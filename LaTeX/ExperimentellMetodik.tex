\documentclass[a4paper,12pt]{article}
\usepackage[swedish]{babel}
\usepackage[utf8]{inputenc}
\usepackage{amsmath, amsthm, amssymb, tikz, cleveref}
\usetikzlibrary{decorations.pathreplacing}
\usepackage[a4paper,includeheadfoot,margin=2.54cm]{geometry}
% Lite formattering, svenska istället för engelska
\crefname{equation}{ekv.}{ekv.}
\crefrangeformat{equation}{ekv.~(#3#1#4) till~(#5#2#6)}
\Crefrangeformat{equation}{Ekv.~(#3#1#4) till~(#5#2#6)}
%
\begin{document}
%
\title{Experimentell Metodik}
%
\author{Zacharias Brohn\thanks{email: \texttt{zacbro-8@student.ltu.se}}\\  
        Elis Bergdahl\thanks{email: \texttt{elieba-4@student.ltu.se}} \\
        Mikael Baer\thanks{email: \texttt{DinMejl}} \\
        \\
        Luleå tekniska universitet \\ 
        971 87 Luleå, Sverige}
%
\date{\today}
%
\maketitle
%
\begin{abstract}
    PH
\end{abstract}
%
\section{Inledning}
    Detta projekt har som syfte att undersöka volymflödet av materia genom smala, horisontella rör. I dessa undersökningar används endast vatten $(H_2O)$ men de matematiska beräkningar och metoder som har använts ska i teorin även vara korrekta för annan materia.
%
\section{Teori}
%
\subsection{Dimensionsanalys}
    En dimensionsanalys är en metod som tillämpas för att kontrollera att framtagna formler inte innehåller felaktiga variabler. Genomförande av metoden innebär att studera variablernas dimensioner på de ingående kvantiteter i formeln.
%
\subsection{Linjäresering}
    För en potensfunktion:
        \begin{align}
            Y = C \cdot x^a
        \end{align}
    Kan exponenten $a$ bestämmas med hjälp av logaritmering i höger- respektive vänsterled,
        \begin{align}
            ln Y = ln C + a \cdot ln x \implies Y = m + k \cdot X
        \end{align}
    Där:
        \begin{align}
            Y = ln y, k = a, X = ln x \text{ och } m = ln C
        \end{align}
    $Y$ plottas mot $X$, lutningen av grafen ges av exponenten $k$ och $m$ är linjens skärningspunkt i $Y$-axeln
%
\section{Dimensionsanalys}
Formeln på sambandet är
\begin{align}
    Q = C \cdot h^\alpha \cdot d^\beta \cdot l^\gamma \cdot \rho^\delta \cdot g^\in \cdot \mu^\varepsilon
\end{align}
där $Q$ är volymflödet, $C$ är konstanten, $h$ är höjden, $d$ är diametern, $l$ är längden, $\rho$ är densiteten, $g$ är tyngdaccelerationen och $\mu$ är viskositeten. Vi vet att
\begin{align}\label{dimQ}
    Q = L^3 \cdot T^{-1} \cdot M^0
\end{align}
\begin{align}
    [Q] &= [C] \cdot [h^\alpha] \cdot [d^\beta] \cdot [l^\gamma] \cdot [\rho^\delta] \cdot [g^\in] \cdot [\mu^\varepsilon]
\end{align}
och enligt resultat från uträkningarna ovan blir exponenterna
%
\begin{align}
    \alpha = 4 \\
    \beta = 1 \\
    \gamma = -1
\end{align}
%
Ekvations system
\begin{align}
    L^3T^{-1} &= L^{\alpha} \cdot L^{\beta} \cdot L^{\gamma} \cdot (M \cdot L^{-3})^{\delta} \cdot (L \cdot T^{-2})^{\in} \cdot (M \cdot L^{-1} \cdot T^{-1})^{\varepsilon} \\
    L^3T^{-1} &= M^{\delta + \in} \cdot L^{\alpha + \beta + \gamma - 3\delta - \in + \varepsilon} \cdot T^{-\in - 2\varepsilon}
\end{align}
så med hjälp av \ref{dimQ} får vi
\begin{align}
    M&: \delta + \in = 0 \label{eq:M}\\
    L&: \alpha + \beta + \gamma - 3\delta - \in + \varepsilon = 3 \label{eq:L}\\
    T&: -\in - 2\varepsilon = -1 \label{eq:T}
\end{align}
och av \cref{eq:M,eq:L,eq:T} kan vi lösa resterande exponenter
\begin{align}
    \delta + \in = 0 \implies \in = - \delta
\end{align}
och om vi substituerar detta i \cref{eq:L} får vi
\begin{align}
    L: \alpha + \beta + \gamma - 3\delta - (-\delta) + \varepsilon = 3
\end{align}
\begin{align*}
    \downarrow
\end{align*}
\begin{align}
    3 &= 4 + 1 - 1 - 2\delta + \varepsilon \\
    3 &= 4 - 2\delta + \varepsilon \\
    -1 &= -2\delta + \varepsilon \\
    \varepsilon &= 2\delta - 1
\end{align}
substituera i \cref{eq:T}
\begin{align}
    -1 &= -(-\delta)-2(2\delta-1) \\
    &= \delta - 4\delta + 2 \\
    &= -3\delta + 2 \\
    -3 &= -3\delta
\end{align}
vilket ger
\begin{align}
    \frac{-3\delta}{-3} = \frac{-3}{-3} \implies \delta = 1 \implies \varepsilon = 2(1)-1 = 1
\end{align}
alltså får vi att ekponenterna är
\begin{align}
    \delta &= 1 \\
    \in &= -1 \\
    \varepsilon &= 1
\end{align}
\end{document}
