\documentclass[a4paper,12pt]{article}
\usepackage[swedish]{babel}
\usepackage[utf8]{inputenc}
\usepackage{amsmath, amsthm, amssymb, tikz}
\usetikzlibrary{decorations.pathreplacing}
\usepackage[a4paper,includeheadfoot,margin=2.54cm]{geometry}
\title{Experimentell Metodik}
%
\author{Zacharias Brohn\thanks{email: \texttt{zacbro-8@student.ltu.se}}\\  
        Elis Bergdahl\thanks{email: \texttt{DinMejl}} \\
        Mikael Baer\thanks{email: \texttt{DinMejl}} \\
        ~ \\
        Luleå tekniska universitet \\ 
        971 87 Luleå, Sverige}
%
\date{\today}
%
\begin{document}
%
\begin{abstract}
    PH
\end{abstract}
%
\section{Dimensionsanalys}
Formeln på sambandet är
\begin{align}
    Q = C \cdot h^\alpha \cdot d^\beta \cdot l^\gamma \cdot \rho^\delta \cdot g^\in \cdot \mu^\varepsilon
\end{align}
resultat är $\alpha = 4$, $\beta = 1$ och $\gamma = -1$.
%
\subsection{Dimensionsanalys}
\begin{align}
    \alpha = 4 \\
    \beta = 1 \\
    \gamma = -1
\end{align}
\begin{align}\label{dimQ}
    Q = L^3 \cdot T^{-1} \cdot M^0
\end{align}
\begin{align}
    [Q] &= [C] \cdot [h^\alpha] \cdot [d^\beta] \cdot [l^\gamma] \cdot [\rho^\delta] \cdot [g^\in] \cdot [\mu^\varepsilon]
\end{align}
Ekvations system
\begin{align}
    L^3T^{-1} &= L^{\alpha} \cdot L^{\beta} \cdot L^{\gamma} \cdot (M \cdot L^{-3})^{\delta} \cdot (L \cdot T^{-2})^{\in} \cdot (M \cdot L^{-1} \cdot T^{-1})^{\varepsilon} \\
    L^3T^{-1} &= M^{\delta + \in} \cdot L^{\alpha + \beta + \gamma - 3\delta - \in + \varepsilon} \cdot T^{-\in - 2\varepsilon}
\end{align}
\begin{align}\label{expMLT}
    M&: \delta + \in = 0 \\
    L&: \alpha + \beta + \gamma - 3\delta - \in + \varepsilon = 3 \\
    T&: -\in - 2\varepsilon = -1
\end{align}
och av \ref{expMLT} får vi
\begin{align}
    \delta + \in = 0 \implies \in = - \delta
\end{align}
och om vi substituerar detta i \ref{dimQ}
\begin{align}
    L: \alpha + \beta + \gamma - 3\delta - (-\delta) + \varepsilon = 3
\end{align}
\begin{align}
    \downarrow
\end{align}
\begin{align}
    3 &= 4 + 1 - 1 - 2\delta + \varepsilon \\
    3 &= 4 - 2\delta + \varepsilon \\
    -1 &= -2\delta + \varepsilon \\
    \varepsilon &= 2\delta - 1
\end{align}
substituera i 11
\begin{align}
    -1 &= -(-\delta)-2(2\delta-1) \\
    &= \delta - 4\delta + 2 \\
    &= -3\delta + 2 \\
    -3 &= -3\delta
\end{align}
\begin{align}
    \downarrow
\end{align}
\begin{align}
    \frac{-3\delta}{-3} = \frac{-3}{-3} \implies \delta = 1 \implies \varepsilon = 2(1)-1 = 1
\end{align}
alltså
\begin{align}
    \delta &= 1 \\
    \in &= -1 \\
    \varepsilon &= 1
\end{align}
\end{document}
