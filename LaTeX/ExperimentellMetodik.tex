\documentclass[a4paper,12pt]{article}
\usepackage[swedish]{babel}
\usepackage[utf8]{inputenc}
\usepackage{amsmath, amsthm, amssymb, tikz, cleveref}
\usetikzlibrary{decorations.pathreplacing}
\usepackage[a4paper,includeheadfoot,margin=2.54cm]{geometry}
% Lite formattering, svenska istället för engelska
\crefname{equation}{ekv.}{ekv.}
\crefrangeformat{equation}{ekv.~(#3#1#4) till~(#5#2#6)}
\Crefrangeformat{equation}{Ekv.~(#3#1#4) till~(#5#2#6)}
%
\begin{document}
%
\title{Experimentell Metodik}
%
\author{Zacharias Brohn\thanks{email: \texttt{zacbro-8@student.ltu.se}}\\  
        Elis Bergdahl\thanks{email: \texttt{elieba-4@student.ltu.se}} \\
        Mikael Baer\thanks{email: \texttt{DinMejl}} \\
        \\
        Luleå tekniska universitet \\ 
        971 87 Luleå, Sverige}
%
\date{\today}
%
\maketitle
%
\begin{abstract}
    Denna rapport presenterar en undersökning av volymflödet genom smala horisontella rör. 
    Genom dimensionsanalys och experimentella metoder studeras sambandet mellan volymflöde 
    och olika fysikaliska parametrar.
\end{abstract}
%
\section{Inledning}
Vi kommer undersöka volymflödet av materia genom smala, horisontella rör. 
Experimenten utförs med vatten ($\mathrm{H_2O}$), men de framtagna matematiska 
modellerna är generellt tillämpbara för andra fluider.
%
\section{Teori}
%
\subsection{Dimensionsanalys}
Dimensionsanalys är en metod för att verifiera matematiska samband genom att 
kontrollera dimensionell konsekvens hos ingående variabler. Metoden är särskilt 
användbar för att validera fysikaliska ekvationer.

\subsection{Linjärisering}
För en potensfunktion av formen:
\begin{equation}
    Y = C \cdot x^a
\end{equation}
kan exponenten $a$ bestämmas genom logaritmering:
\begin{equation}
    \ln Y = \ln C + a \cdot \ln x \equiv Y' = m + k \cdot X
\end{equation}
där:
\begin{equation}
    Y' = \ln y,\quad k = a,\quad X = \ln x,\quad m = \ln C
\end{equation}
%
\section{Dimensionsanalys av Volymflöde}
Det generella sambandet för volymflödet ges av:
\begin{equation}
    Q = C \cdot d^\alpha \cdot h^\beta \cdot l^\gamma \cdot \rho^\delta \cdot g^\epsilon \cdot \mu^\varepsilon
    \label{eq:general_flow}
\end{equation}
%
där variablerna har följande dimensioner:
\begin{itemize}
    \item $Q$ är volymflödet [$\mathrm{L^3T^{-1}}$]
    \item $d$ är rörets diameter [$\mathrm{L}$]
    \item $h$ är höjdskillnaden [$\mathrm{L}$]
    \item $l$ är rörets längd [$\mathrm{L}$]
    \item $\rho$ är vätskans densitet [$\mathrm{ML^{-3}}$]
    \item $g$ är tyngdaccelerationen [$\mathrm{LT^{-2}}$]
    \item $\mu$ är vätskans viskositet [$\mathrm{ML^{-1}T^{-1}}$]
\end{itemize}
%
Från dimensionsanalysen vet vi att:
\begin{equation}
    [Q] = \mathrm{L^3T^{-1}M^0}
    \label{eq:dim_Q}
\end{equation}
%
Dimensionell analys ger:
\begin{equation}
    [Q] = [C] \cdot [d^\alpha] \cdot [h^\beta] \cdot [l^\gamma] \cdot [\rho^\delta] \cdot [g^\epsilon] \cdot [\mu^\varepsilon]
    \label{eq:dim_analysis}
\end{equation}
%
Från tidigare beräkningar har vi fått:
\begin{align}
    \alpha &= 4 \label{eq:alpha} \\
    \beta &= 1 \label{eq:beta} \\
    \gamma &= -1 \label{eq:gamma}
\end{align}
%
Det resulterande ekvationssystemet blir:
\begin{align}
    \mathrm{L^3T^{-1}} &= \mathrm{L^{\alpha + \beta + \gamma}} \cdot 
    \mathrm{(ML^{-3})^\delta} \cdot \mathrm{(LT^{-2})^\epsilon} \cdot 
    \mathrm{(ML^{-1}T^{-1})^\varepsilon} \\
    \mathrm{L^3T^{-1}} &= \mathrm{M^{\delta + \varepsilon}} \cdot 
    \mathrm{L^{\alpha + \beta + \gamma - 3\delta + \epsilon - \varepsilon}} \cdot 
    \mathrm{T^{-2\epsilon - \varepsilon}}
\end{align}
%
Genom att jämföra exponenter får vi:
\begin{align}
    \mathrm{M}&: \delta + \varepsilon = 0 \label{eq:M_balance} \\
    \mathrm{L}&: \alpha + \beta + \gamma - 3\delta + \epsilon - \varepsilon = 3 \label{eq:L_balance} \\
    \mathrm{T}&: -2\epsilon - \varepsilon = -1 \label{eq:T_balance}
\end{align}
%
Ur \cref{eq:M_balance} får vi:
\begin{equation}
    \epsilon = -\delta \label{eq:epsilon}
\end{equation}
%
Substitution i \cref{eq:L_balance} ger:
\begin{align}
    4 + 1 - 1 - 3\delta - \delta + \varepsilon &= 3 \\
    4 - 4\delta + \varepsilon &= 3 \\
    \varepsilon &= 2\delta - 1
\end{align}
%
från \cref{eq:T_balance}:
\begin{align}
    -1 &= -(-\delta) - 2(2\delta - 1) \\
    -1 &= \delta - 4\delta + 2 \\
    -3 &= -3\delta \\
    \delta &= 1
\end{align}
%
från det kan vi lösa \cref{eq:epsilon}
\begin{align}
    \epsilon = -\delta = -1
\end{align}
alltså får vi att ekponenterna är
\begin{align}
    \delta &= 1 \\
    \epsilon &= -1 \\
    \varepsilon &= 2(1) - 1 = 1
\end{align}
\end{document}
