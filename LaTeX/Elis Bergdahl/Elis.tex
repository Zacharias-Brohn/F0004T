\documentclass[a4paper,12pt]{article}
\usepackage[english]{babel}
\usepackage[utf8]{inputenc}
\usepackage{amsmath, amsthm, amssymb, tikz}
\usetikzlibrary{decorations.pathreplacing}
\usepackage[a4paper,includeheadfoot,margin=2.54cm]{geometry}
\title{Experimentell Metodik}
%
\author{Zacharias Brohn\thanks{email: \texttt{zacbro-8@student.ltu.se}}\\  
        Elis Bergdahl\thanks{email: \texttt{elieba-4@student.ltu.sel}} \\
        Mikael Baer\thanks{email: \texttt{DinMejl}} \\
        ~ \\
        Luleå tekniska universitet \\ 
        971 87 Luleå, Sverige}
%
\date{\today}
%
\begin{document}
%
\maketitle
%
\begin{abstract}
\end{abstract}
%
\section{Inledning}
    Detta projekt har som syfte att undersöka volymflödet av materia genom smala, horisontella rör. I dessa undersökningar används endast vatten $(H_2O)$ men de matematiska beräkningar och metoder som har använts ska i teorin även vara korrekta för annan materia.
%
\section{Teori}
%
\subsection{Dimensionsanalys}
    En dimensionsanalys är en metod som tillämpas för att kontrollera att framtagna formler inte innehåller felaktiga variabler. Genomförande av metoden innebär att studera variablernas dimensioner på de ingående kvantiteter i formeln.
%
\subsection{Linjäresering}
    För en potensfunktion:
        \begin{align}
            Y = C \cdot x^a
        \end{align}
    Kan exponenten $a$ bestämmas med hjälp av logaritmering i höger- respektive vänsterled,
        \begin{align}
            ln Y = ln C + a \cdot ln x \implies Y = m + k \cdot X
        \end{align}
    Där:
        \begin{align}
            Y = ln y, k = a, X = ln x \text{ och } m = ln C
        \end{align}
    $Y$ plottas mot $X$, lutningen av grafen ges av exponenten $k$ och $m$ är linjens skärningspunkt i $Y$-axeln
%
\end{document}
